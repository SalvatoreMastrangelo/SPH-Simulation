\section{Equazioni Governanti}
    L'interazione tra particelleèdi tipo locale, e si estende fino ad una distanza
    prefissata chiamata \textit{Smoothing Radius}. L'intensitá dell'interazione
    dipende da una funzione di smoothing, chiamata \textit{Kernel Function}, che
   èuna funzione ad area unitaria e decrescente con la distanza. Spesso é
    definita come:
    \[W(|\mathbf{r} - \mathbf{r'}|, h)\]
    dove:
    \begin{itemize}
        \item $\mathbf{r}, \mathbf{r'}$: posizioni delle particelle
        \item $h$: smoothing radius
    \end{itemize}
    Tutte le grandeezze fisiche, come densitá, e forze di pressione, sono calcolate
    come: 
    \[
    A(\mathbf{r}) = \int A(\mathbf{r}') \, W\left(| \mathbf{r} - \mathbf{r}' |, h\right) \, dV(\mathbf{r}')
    \]
    che puó essere discretizzata come:
    \[
    A(\mathbf{r}) = \sum_{j} V_j A_j \, W\left(| \mathbf{r} - \mathbf{r}_j |, h\right)
    \]

    dove:
    \begin{itemize}
        \item $A(\mathbf{r})$ rappresenta tale grandezza fisica in un punto $\mathbf{r}$.
        \item $V_j$èil volume associato alla particella $j$.
        \item $A_j$èil valore della grandezza fisica per la particella $j$.
    \end{itemize}  
    
    \newpage

    Ad esempio, per il calcolo della densitá:
    \[\rho(\mathbf{r}) = \sum_{j} V_j \rho_j \, W\left(| \mathbf{r} - \mathbf{r}_j |, h\right) = \sum_{j} m_j \, W\left(| \mathbf{r} - \mathbf{r}_j |, h\right)\] 
