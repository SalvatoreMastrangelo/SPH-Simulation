\section{Possibili Sviluppi Futuri}
    Possibili Sviluppi sono articolabili sia riguardo l'accuratezza fisica sia riguardo la 
    performance della simulazione.

    \subsection{Tensione Superficiale}
        La simulazione attuale non include fenomeni di tensione superficiale, 
        modellizzabili come forze attrattive tra particelle vicine. L'attrazione in questione 
        vedrebbe coinvolta una terza kernel function che impone una forza attrattiva tra particelle,
        che prevale a distanze maggiori rispetto alla forza di pressione, e che é trascurabile a 
        distanze ravvicinate, dove prevale l'ingombro della particella rappresentato come forza di 
        pressione.

    \subsection{Parallelizzazione mediante GPU}
        L'attuale implementazione prevede solo una parallelizzazione a livello CPU per il calcolo
        delle grandezze fisiche, ma essendo la task fortemente parallelizzabile, sarebbe possibile 
        sviluppare dei kernel per GPU. Ciononostante, questa idea richiederebbe una riscrittura della
        funzione di ricerca del neighborhood, che andrebbe ad utilizzare strutture dati 
        piú adatte e tecniche di calcolo di hash per ciascuna cella.

    \subsection{Trasposizione in 3D}
        Pur essendo implementata in 2D, la trasposizione in 3 dimensioni sarebbe relativamente
        semplice, richiedendo solo una riscrittura delle kernel functions e delle funzioni 
        di supporto di calcolo tra vettori. Un problema piú complesso potrebbe porsi nell'utilizzo
        della griglia per i neighbor, che in 3D risulterebbe molto piú dispendiosa in termini di memoria.